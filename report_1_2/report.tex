\documentclass[bachelor, och, labwork]{SCWorks}
% параметр - тип обучения - одно из значений:
%    spec     - специальность
%    bachelor - бакалавриат (по умолчанию)
%    master   - магистратура
% параметр - форма обучения - одно из значений:
%    och   - очное (по умолчанию)
%    zaoch - заочное
% параметр - тип работы - одно из значений:
%    referat    - реферат
%    coursework - курсовая работа (по умолчанию)
%    diploma    - дипломная работа
%    pract      - отчет по практике
% параметр - включение шрифта
%    times    - включение шрифта Times New Roman (если установлен)
%               по умолчанию выключен
\usepackage{subfigure}
\usepackage{tikz,pgfplots}
\pgfplotsset{compat=1.5}
\usepackage{float}

%\usepackage{titlesec}
\setcounter{secnumdepth}{4}
%\titleformat{\paragraph}
%{\normalfont\normalsize}{\theparagraph}{1em}{}
%\titlespacing*{\paragraph}
%{35.5pt}{3.25ex plus 1ex minus .2ex}{1.5ex plus .2ex}

\titleformat{\paragraph}[block]
{\hspace{1.25cm}\normalfont}
{\theparagraph}{1ex}{}
\titlespacing{\paragraph}
{0cm}{2ex plus 1ex minus .2ex}{.4ex plus.2ex}

% --------------------------------------------------------------------------%
\usepackage[T2A]{fontenc}
\usepackage[utf8]{inputenc}
\usepackage{graphicx}
\graphicspath{ {./img/} }
\usepackage{tempora}

\usepackage[sort,compress]{cite}
\usepackage{amsmath}
\usepackage{amssymb}
\usepackage{amsthm}
\usepackage{fancyvrb}
\usepackage{listings}
\usepackage{listingsutf8}
\usepackage{longtable}
\usepackage{array}
\usepackage[english,russian]{babel}

\usepackage[colorlinks=true, linkcolor=black]{hyperref}
\usepackage{url}

\usepackage{underscore}
\usepackage{setspace}
\usepackage{indentfirst} 
\usepackage{mathtools}
\usepackage{amsfonts}
\usepackage{enumitem}
\usepackage{tikz}

\usepackage{minted}
\setminted[python3]{style=bw, linenos, breaklines=true, fontsize=\footnotesize}

\newcommand{\eqdef}{\stackrel {\rm def}{=}}
\newcommand{\specialcell}[2][c]{%
\begin{tabular}[#1]{@{}c@{}}#2\end{tabular}}

\renewcommand\theFancyVerbLine{\small\arabic{FancyVerbLine}}

\newtheorem{lem}{Лемма}

\begin{document}

% Кафедра (в родительном падеже)
\chair{теоретических основ компьютерной безопасности и криптографии}

% Тема работы
\title{Теория псевдослучайных генераторов}

% Курс
\course{4}

% Группа
\group{431}

% Факультет (в родительном падеже) (по умолчанию "факультета КНиИТ")
\department{факультета КНиИТ}

% Специальность/направление код - наименование
%\napravlenie{09.03.04 "--- Программная инженерия}
%\napravlenie{010500 "--- Математическое обеспечение и администрирование информационных систем}
%\napravlenie{230100 "--- Информатика и вычислительная техника}
%\napravlenie{231000 "--- Программная инженерия}
\napravlenie{10.05.01 "--- Компьютерная безопасность}

% Для студентки. Для работы студента следующая команда не нужна.
% \studenttitle{Студентки}

% Фамилия, имя, отчество в родительном падеже
\author{Стаина Романа Игоревича}

% Заведующий кафедрой
\chtitle{} % степень, звание
\chname{}

%Научный руководитель (для реферата преподаватель проверяющий работу)
\satitle{доцент} %должность, степень, звание
\saname{И.~И.~Слеповичев}

% Руководитель практики от организации (только для практики,
% для остальных типов работ не используется)
% \patitle{к.ф.-м.н.}
% \paname{С.~В.~Миронов}

% Семестр (только для практики, для остальных
% типов работ не используется)
%\term{8}

% Наименование практики (только для практики, для остальных
% типов работ не используется)
%\practtype{преддипломная}

% Продолжительность практики (количество недель) (только для практики,
% для остальных типов работ не используется)
%\duration{4}

% Даты начала и окончания практики (только для практики, для остальных
% типов работ не используется)
%\practStart{30.04.2019}
%\practFinish{27.05.2019}

% Год выполнения отчета
\date{2023}

\maketitle

% Включение нумерации рисунков, формул и таблиц по разделам
% (по умолчанию - нумерация сквозная)
% (допускается оба вида нумерации)
% \secNumbering

%-------------------------------------------------------------------------------------------

% \begin{minted}[fontsize=\small]{MySQL}
% \end{minted}

% \begin{figure}[H]
%     \centering
%     \includegraphics[width=0.999\textwidth]{img/}
%     \caption{}
%     \label{easy_hack}
% \end{figure}

\tableofcontents

\section{Генератор псевдослучайных чисел}
Создайте программу для генерации псевдослучайных величин следующими алгоритмами:
\begin{enumerate}
  \item Линейный конгруэнтный метод;
  \item Аддитивный метод;
  \item Пятипараметрический метод;
  \item Регистр сдвига с обратной связью (РСЛОС);
  \item Нелинейная комбинация РСЛОС;
  \item Вихрь Мерсенна;
  \item RC4;
  \item ГПСЧ на основе RSA;
  \item Алгоритм Блюма-Блюма-Шуба.
\end{enumerate}

\subsection{Линейный конгруэнтный метод}
Последовательность ПСЧ, получаемая по формуле:
\[ X_{n + 1} = (aX_n + c) \mod m, \; n \geq 1, \]
называется \textit{линейной конгруэнтной последовательностью (ЛКП)}.
Параметры:
\begin{itemize}
  \item $m > 0$, модуль;
  \item $0 \leq a \leq m$, множитель;
  \item $0 \leq c \leq m$, приращение;
  \item $0 \leq X_0 \leq m$, начальное значение.
\end{itemize}

При запуске программы дополнительно проверяется, что выполняются следующие условия, при выполнении которых ЛКП имеет период $m$:
\begin{enumerate}
  \item числа $c$ и $m$ взаимно простые;
  \item $a - 1$ кратно $p$ для некоторого простого $p$, являющегося делителем $m$;
  \item $a - 1$ кратно 4, если $m$ кратно 4.
\end{enumerate}

\subsection{Аддитивный метод}
Последовательность определяется следующим образом:
\[ X_n = (X_{n - k} + X_{n - j}) \mod m, \; j > k \geq 1 \].
Параметры:
\begin{itemize}
  \item $m > 0$, модуль;
  \item $k$, младший индекс;
  \item $j$, старший индекс;
  \item последовательность из $j$ начальных значений.
\end{itemize}

\subsection{Пятипараметрический метод}
Данный метод является частным случаем РСЛОС, использует характеристический многочлен из 5 членов и позволяет генерировать последовательности $w$-битовых двоичных целых чисел в соответствии со следующей рекуррентной 
формулой:
\[ X_{n + p} = X_{n + q_1} + X_{n + q_2} + X_{n + q_3} + X_n, \; n = 1, 2, 3, \dots \]
Параметры:
\begin{itemize}
  \item $p$;
  \item $q_1, q_2, q_3$;
  \item $w$;
  \item начальное значение.
\end{itemize}

\subsection{Регистр сдвига с обратной связью (РСЛОС)}
Регистр сдвига с обратной линейной связью (РСЛОС) "--- регистр сдвига битовых слов, у которого входной (вдвигаемый) бит является линейной функцией остальных битов. Вдвигаемый вычисленный бит заносится в 
ячейку с номером 0. Количество ячеек $p$ называют длиной регистра.

Одна итерация алгоритма, генерирующего последовательность, состоит 
из следующих шагов:
\begin{enumerate}
  \item Содержимое ячейки $p - 1$ формирует очередной бит ПСП битов.
  \item Содержимое ячейки 0 определяется значением функции обратной связи, 
  являющейся линейной булевой функцией с коэффициентами $a_1, \dots, a_{p - 1}$.
  \item Содержимое каждого $i$-го бита перемещается в $(i + 1)$-й, $0 \leq i < p - 1$. 
  \item В ячейку 0 записывается новое содержимое, вычисленное на шаге 2.
\end{enumerate}

Параметры:
\begin{itemize}
  \item двоичное представление вектора коэффициентов;
  \item начальное значение регистра.
\end{itemize}

\subsection{Нелинейная комбинация РСЛОС}
Последовательность получается из нелинейной комбинации трёх РСЛОС следующим образом:
\[ f(x_1, x_2, x_3) = x_1 x_2 \oplus x_2 x_3 \oplus x_3 \]

Параметры:
\begin{itemize}
  \item двоичное представление вектора коэффициентов для $R1, R2, R3$;
  \item $x_1, x_2, x_3$ "--- десятичное представление начальных состояний регистров $R1, R2, R3$;
  \item $w$, длина слова.
\end{itemize}

\subsection{Вихрь Мерсенна}
Следующий псевдокод представляет алгоритм генерации ПСЧ:
\begin{minted}[fontsize=\footnotesize]{text}
  // Создание массива длины n для сохранения состояний генератора
  int[0..n-1] MT
  int index := n+1
  const int lower_mask = (1 << r) - 1
  const int upper_mask = lowest w bits of (not lower_mask)
  
  // Initialize the generator from a seed
  function seed_mt(int seed) {
      index := n
      MT[0] := seed
      for i from 1 to (n - 1) { // loop over each element
          MT[i] := lowest w bits of (f * (MT[i-1] xor (MT[i-1] >> (w-2))) + i)
      }
  }
  
  // Извлечение чисел на основе массива MT[index]
  // Вызывает twist() каждые n чисел
  function extract_number() {
      if index >= n {
          twist()
      }
  
      int y := MT[index]
      y := y xor ((y >> u) and d)
      y := y xor ((y << s) and b)
      y := y xor ((y << t) and c)
      y := y xor (y >> l)
  
      index := index + 1
      return lowest w bits of (y)
  }
  
  // Генерация следующих n значений
  function twist() {
      for i from 0 to (n-1) {
          int x := (MT[i] and upper_mask)
                    | (MT[(i+1) mod n] and lower_mask)
          int xA := x >> 1
          if (x mod 2) != 0 { // lowest bit of x is 1
              xA := xA xor a
          }
          MT[i] := MT[(i + m) mod n] xor xA
      }
      index := 0
  }  
\end{minted}

Константы, используемые в алгоритме:
\begin{itemize}
  \item $p = 624$;
  \item $w = 32$;
  \item $r = 31$;
  \item $q = 397$;
  \item $a = 2567483615$;
  \item $u = 11$;
  \item $s = 7$;
  \item $t = 15$;
  \item $l = 18$;
  \item $b = 2636928640$;
  \item $c = 4022730752$.
\end{itemize}

Параметры:
\begin{itemize}
  \item модуль;
  \item начальное значение.
\end{itemize}

\subsection{RC4}
Описание алгоритма:
\begin{enumerate}
  \item Инициализация $S_i$.
  \item $i = 0, j = 0$.
  \item Итерация алгоритма:
    \begin{enumerate}
      \item $i = (i + 1) \mod 256$;
      \item $j = (j + S_i) \mod 256$;
      \item $Swap(S_i, S_j)$;
      \item $t = (S_i + S_j) \mod 256$;
      \item $K = S_t$.
    \end{enumerate}
\end{enumerate}

Параметры:
\begin{itemize}
  \item 256 начальных значений $S_i$.
\end{itemize}

\subsection{ГПСЧ на основе RSA}
Описание алгоритма:
\begin{enumerate}
  \item Инициализация чисел: $n = pq$, где $p$ и $q$ простые числа, числа $e$: $1 < e < f$, НОД($e, f$) = 1, $f = (p - 1)(q - 1)$ и числа $x_0$ из интервала $[1, n - 1]$.
  \item \texttt{For i = 1 to w do}
        \begin{enumerate}
          \item $x_i \leftarrow x_{i-1}^e \mod n$.
          \item $z_i \leftarrow $ последний значащий бит $x_i$
        \end{enumerate}
  \item Вернуть $z_1, \dots, z_w$.
\end{enumerate}

Параметры:
\begin{itemize}
  \item $n$, модуль;
  \item $e$, число;
  \item $w$, длина слова;
  \item $x_0$, начальное значение.
\end{itemize}

\subsection{Алгоритм Блюм-Блюма-Шуба}
Описание алгоритма:
\begin{enumerate}
  \item Инициализация числа: $n = 127 \cdot 131 = 16637$.
  \item Вычислим $x_0 = x^2 \mod n$, которое будет начальным вектором.
  \item \texttt{For i = 1 to w do}
        \begin{enumerate}
          \item $x_i \leftarrow x_{i-1}^2 \mod n$.
          \item $z_i \leftarrow $ последний значащий бит $x_i$
        \end{enumerate}
  \item Вернуть $z_1, \dots, z_w$.
\end{enumerate}

Параметры:
\begin{itemize}
  \item $n$, модуль;
\end{itemize}

\section{Преобразование ПСЧ к заданному распределению}
Создать программу для преобразования последовательности ПСЧ в другую последовательность
ПСЧ с заданным распределением:
\begin{enumerate}
  \item Стандартное равномерное с заданным интервалом;
  \item Треугольное распределение;
  \item Общее экспоненциальное распределение;
  \item Нормальное распределение;
  \item Гамма распределение;
  \item Логнормальное распределение;
  \item Логистическое распределение;
  \item Биномиальное распределение.
\end{enumerate}

\subsection{Метод генерации случайной величины}
Если максимальное значение равномерного целого случайного числа $X$
равно $(m - 1)$, для генерации стандартных равномерных случайных чисел 
необходимо применять следующую формулу: $U = X / m$.

\subsection{Стандартное равномерное с заданным интервалом}
\[ Y = bU + a \]

\subsection{Треугольное распределение}
\[ Y = a + b(U_1 + U_2 - 1) \]

\subsection{Общее экспоненциальное распределение}
\[ Y = -b \ln(U) + a \]

\subsection{Нормальное распределение}
\[ Z_1 = \mu \sigma \sqrt{-2 \ln(1 - U_1)} \cos(2 \pi U_2) \]
\[ Z_2 = \mu \sigma \sqrt{-2 \ln(1 - U_1)} \sin(2 \pi U_2) \]

\subsection{Гамма распределение}
Алгоритм для $c = k$ ($k$ -- целое число)
\[ Y = a - b \ln \{(1 - U_1) \dots (1 - U_k)\} \]

\subsection{Логнормальное распределение}
\[ Y = a + \exp(b - Z) \]

\subsection{Логистическое распределение}
\[ Y = a + b \ln(\frac{U}{1 - U})\]

\subsection{Биномиальное распределение}
Следующий псевдокод представляет алгоритм преоб
\begin{minted}[fontsize=\footnotesize]{text}
y = binominal(x, a, b, m):
  u = U(x)
  s = 0
  k = 0
  loopstart:
    s = s + C(k, b) + a^k * (1 - a)^(b - k)
    if s > u:
      y = k
      Завершить
    if k < b - 1:
      k = k + 1
      move to loopstart
    y = b
\end{minted}
\newpage
\appendix
    \section{Код задания 1}
    \inputminted[fontsize=\footnotesize]{text}{main.cpp}

    \section{Код задания 2}
    \inputminted[fontsize=\footnotesize]{text}{main2.cpp}


%     \section{Код \texttt{Bit.py}}
%     \inputminted[fontsize=\footnotesize]{text}{code/Bit.py}

%     \section{Код \texttt{HuffmanEncode.py}}
%     \inputminted[fontsize=\footnotesize]{text}{code/HuffmanEncode.py}

%     \section{Код \texttt{HuffmanDecode.py}}
%     \inputminted[fontsize=\footnotesize]{text}{code/HuffmanDecode.py}


\end{document}